\documentclass{extarticle}

\usepackage{listings} % To include source code
\usepackage{xspace} % To avoid problems with missing spaces after predefined symbols    

\author{Diego R.R.}
\title{Word Search Program Notes}
\date{\today}

\newcommand{\code}[1]{$#1$} % command to write c++ code in text
\newcommand{\prgmtitle}{\textit{word-search}\xspace} % predefined program title

\begin{document}
\maketitle

\section{Program Description}
The \prgmtitle is a program in c++ that solves the problem of searching
words in a soup of characters. The \textit{soup} is defined as a matrix
of characters of size $N \times M$, or equivalently in c++, \code{char[N][M]}. The words
to be searched are movie titles given as a vector of strings: \code{vector<strings> movies\_in\_soup}. 
Then the words are searched in the soup and the program will know which movies are in the soup and which are not.

\section{Program Structure}
\prgmtitle will be divided in the following parts:
    \begin{enumerate}
        \item The input loader
        \item The parser.
        \item The solver.
        \item The output handler.
    \end{enumerate}

\subsection{Input Loader}
The input loader will define how input is handled and introduced.
For the moment the case considered would be the input given as a \code{string}
in \code{cin}

\section{Input File Format}
The word-search program will have as input .txt files and potentially user-typed movie titles.
But in priory, the input will be a .txt file given as a \code{string}. Input files are assumed
to be already included in the same directory as the program. The format of the .txt file
will be as follows:
    \begin{enumerate}
        \item The first block is a single line with a pair of numbers $N$ and $M$ separated by a space. 
        $N$ is the number of rows and $M$ is the number of columns.
        \item The second block is a matrix of $N$ rows and $M$ columns with the letters of the word search.
        \item The third block is a list of words to be searched in the matrix.
    \end{enumerate}
Now consider that the file may have comment lines anywhere including
inside the blocks of inputs. Comment lines should be treated as if the 
parser had never seen them.

I will define the parser to be a set of functions that will take
as input the .txt file and will return all the useful information
interpretet as c++ data types. Such information could be:
    \begin{enumerate}
        \item The entry size of the matrix $N$ and $M$.
        \item The declaration of the characters of the matrix. 
        \item The list of words to be searched.
    \end{enumerate}
\end{document}